February 5, 2023
---
My Solutions to Exercise 9.5 of Brian Hall's Lie Theory Book
---

Chapter 9, Exercise 5: Let $\mu$ be any element of $\mathfrak{h}$ and let $W_\mu = U(\mathfrak{g})/I_\mu$ be the Verma module with
highest weight $\mu$. Now let $\sigma_\mu$ be any other highest weight cyclic representation of $\mathfrak{g}$ with highest weight $\mu$, acting on a vector space $V_\mu$. Show that there is a surjective intertwining map $\phi$ of $W_\mu$ onto $V_\mu$.\\
Note: It follows that $V_\mu$ is isomorphic to the quotient space $W_\mu/\ker \phi$. Thus, the Verma module is maximal among highest weight cyclic representations with highest weight $\mu$, in the sense that every other such representation is a quotient of $W_\mu.$\\
\emph{Hint: } If $\tilde{\sigma_\mu}$ is the extension of $\sigma_\mu$ to $U(\mathfrak{g})$, as in Proposition 9.9, construct a map $\psi: U(\mathfrak{g}) \to V_\mu$ by mapping $\alpha \in U(\mathfrak{g})$ to $\tilde{\sigma_\mu}(\alpha)w_0$ where $w_0$ is a highest weight vector of $V_\mu$.
Following the hint, one can show that $\ker \psi \supset I_\mu$, so we have a map $\overline{\psi}:W_\mu = U(\mathfrak{g})/I_\mu \to V_\mu.$ This map is onto because $V_\mu$ will be generated by the lowering operators applied to the highest weight vector $w_0$. It is also an intertwining map because $$\tilde{\sigma_\mu}(X)\overline{\psi}(Y) = \tilde{\sigma_\mu}(X)(\tilde{\sigma_\mu}(Y)w_0) = \tilde{\sigma_\mu}(XY)w_0 = \overline{\psi}(XY) = \bar{\psi}(\pi_\mu(X)Y),$$
where $X\in \mathfrak{g}$ and $Y\in W_\mu$. Therefore, it is the desired map in the problem.